\documentclass[12pt]{article}
\usepackage[utf8]{inputenc}
\usepackage[english,russian]{babel}
\usepackage[T2A]{fontenc}
\usepackage{amsmath}
\usepackage{amssymb}
\usepackage{amsfonts}
\usepackage{xcolor}
\usepackage{enumitem}
\usepackage[top=2cm, bottom=2cm, left=2cm, right=2.5cm]{geometry}
\usepackage{lastpage}
\usepackage{fancyhdr}
\usepackage{mathrsfs}
\usepackage{multicol}
\usepackage[hidelinks]{hyperref}
\usepackage{tikz}
\usepackage{wasysym}
\usepackage{amsmath}
\usepackage[most]{tcolorbox}
\usepackage{parskip}
\usepackage{float}
\restylefloat{table}
\graphicspath{{images/}}


\usepackage{fontspec}
\setmainfont{Helvetica}
% \setmainfont{CMU Bright}
% \setmainfont{CMU Serif}

\hypersetup{
    colorlinks=true,
    linkcolor=cyan, % blue
    filecolor=magenta,      
    urlcolor=cyan,
    pdftitle={HW1 Dmitry Uspenskiy},
    pdfpagemode=FullScreen,
}

\newcommand{\imgh}[3]
{
\begin{figure}[h]
\center{\includegraphics[width=#1]{#2}}
\caption{#3}
\label{ris:#2}
\end{figure}
}

\newcommand{\condition}[1]
{
\begin{tcolorbox}[enhanced jigsaw,
    sharp corners,
    boxrule=0.5pt, 
    colback=white!30!white,   
    borderline={0.5pt}{-2pt}{black,solid} % 0.5pt linewith, -2pt outside, black solid linestyle
]
#1
\end{tcolorbox}
}

% \setcounter{section}{-1}
\hyphenpenalty=10000

\pagestyle{fancyplain}
\headheight 35pt
\rhead{\textbf{Выполнил:} Успенский Д. А. \\ \textbf{Группа:} 208}
\chead{\textbf{\large ДЗ 1} \\ [3ex] }
\lhead{ФКН ВШЭ \\ Process Mining \\ Осенний семестр 2023} 
\lfoot{}
\cfoot{}
\rfoot{\small\thepage}
\headsep 3em

\begin{document}
\tableofcontents
\newpage

\section{Проблемы и задачи}
% \subsection*{Проблемы}
\begin{itemize}
    \item Не все счета успешно доходят до оплаты. Необходимо выяснить, где они останавливаются и почему.

    \item Определить лишние операции в процессе, определить топ сотрудников, их совершающих и время, затраченное на эти действия.

    \item В процессе производятся излишние (повторные проверки). Определить причины повторов и излишние затраты времени на них.
\end{itemize}
\textbf{Дедлайн: 24 октября, 9:30.}
\newpage

\section{Описание датасета}
Собраны данные по оплатах примерно за два года.

Счета бывают двух типов: оплата возмещения сотруднику и счёт перед юр. лицом.

\begin{itemize}
    \item \textbf{'Экземпляр'} - id счёта

    \item \textbf{'Операция'} - тип операции, проведённой над счётом. Всего 15 типов.

    \item \textbf{'Дата начала'} - дата начала операции. Самая ранняя: 2021-01-07 02:49:12', поздняя: '2022-12-06 23:42:17'

    \item \textbf{'Тип операции'} - тип операции: ['Получение', 'Согласование', 'Платеж', 'Сбор данных']

    \item \textbf{'Стоимость активности'} - стоимость операции

    \item \textbf{'Сумма'} - сумма счёта

    \item \textbf{'Автоматизация'} - тип операции: ['Автоматизированная', 'Ручная']

    \item \textbf{'Ответственный'} - ФИО сотрудника

    \item \textbf{'Статус счета'} - статус: ['Закрыт', 'В процессе']

    \item \textbf{'Тип счета'} - тип счёта по размеру и категории поставщика: ['Счет на малую сумму', 'Счет на среднюю сумму', 'Доверенный поставщик', 'Авансовый отчет', 'Партнер', 'Счет на большую сумму']

    \item \textbf{'Регион'} - регион принятия решения по данному типу конкретной операции

    \item \textbf{'Клиент'} - ФИО клиента

    \item \textbf{'Тип клиента'} - тип клиента: ['Обычный клиент', 'Импульсивный клиент', 'Клиент со скидкой', 'Внутренний клиент', 'Партнер']

    \item \textbf{'Поставщик'} - юр. имя поставщика

    \item \textbf{'Тип поставщика'} - тип поставщика (в основном 'Производственное оборудование'): ['Производственное оборудование', 'Инженерное оборудование', 'Сотрудники', 'Офисные принадлежности']

    \item \textbf{'Отдел'} - отдел принятия решения по данному типу конкретной операции, в основном 'Внутренний контроль'

    \item \textbf{'Пользователь'} - тот, кто принимал решение на данном этапе оплаты счёта

    \item \textbf{'Путь'} - 357 уникальных путей. Это тип цепочки событий процесса.

    \item \textbf{'Тип пути'} - алгоритм выплаты счёта
\end{itemize}

\begin{table}[H]
    \begin{tabular}{llllc}
    \hline
    Экземпляр &                    Операция &         Дата начала & Тип операции & \dots \\
    \hline
    IF-1506417 &          A. Получение счета & 2021-06-27 10:00:23 &    Получение & \dots \\
    IF-1506417 &           C. Проверка счета & 2021-06-27 10:00:25 &    Получение & \dots \\
    IF-1506417 & G. Финальная проверка счета & 2021-06-27 10:41:50 & Согласование & \dots \\
    IF-1506417 &             H. Согласование & 2021-06-28 10:10:55 & Согласование & \dots \\
    IF-1506417 &             N. Оплата счета & 2021-07-01 17:01:47 &       Платеж & \dots \\
    \hline
    \end{tabular}
\caption{Пример лога для чека}
\end{table}

По обработка заявки заняла 4 дня и 7 часов. В обработке участвовало 3 сотрудника.

Обработка одного платежа имеет множество этапов, проходит согласование у разных сотрудников на разных уровнях и в разных регионах. В данной ситуации различных комбинаций путей заявки слишком много, так что задачу оптимизации не решить методом "пристального взгляда"\ на гистограмму.


\imgh{19cm}{metric_graph.png}{Граф популярных путей обработки чека.}
\newpage

\section{Поиск причин неоплаченных заявок}
Для нахождения слабого звена в процессе оплаты счетов использовалась библиотека SberPM.

Рассмотрим цепочки, которые не завершились оплатой.
\imgh{17cm}{not_paid.png}{Цепочки, не дошедшие до оплаты.}

Видим повторяющиеся фамилии. На данном этапе есть две гипотезы:
\begin{itemize}
    \item сотрудники систематически не закрывают оплату счёта
    \item через этих сотрудников проходит значительный поток платежей, в том числе и не оплаченных
\end{itemize}

Для дальнейшего расследования рассмотрим самый популярный сценарий обработки заявки, не дошедший до платежа.: \textit{A. Получение счета} $\rightarrow$ \textit{C. Проверка счета}.

На всех заявках не был назначен ответственный. 
Заявки останавливались на данных сотрудниках:

\begin{small}
\begin{tabular}{lr}
    \hline
    Пользователь                   &    Количество \\
    \hline
    Система                        &           529 \\
    Кондрат Григорьевич Ковалев    &           525 \\
    Макаров Измаил Брониславович   &             2 \\
    Эмилия Константиновна Суворова &             1 \\
    Нинель Антоновна Кудряшова     &             1 \\
    \hline
\end{tabular} 
\end{small}
\newpage

Разберёмся, как в принципе проходили платежи через сотрудника \textit{Кондрат Григорьевич Ковалев}. Для этого соберём все счета, проходившие через него и визуализируем на графе со временными и количественными метриками. 

\imgh{13cm}{suspicious_metric_graph.png}{Работа подозрительного сотрудника.}

\subsection*{Выводы}
По какой-то причине руководитель \textit{Кондрат Григорьевич Ковалев} не назначал ответственного за оплату и счета останавливались на этапе проверки или финального согласования. 
\newpage


\section{Поиск лишних операций}
Для выявления лишних операций и сотрудников, которые их выполняют предлагается выполнить следующие действия:

\begin{enumerate}
    \item Выделить признаки, характеризующие каждый чек (объект) на всех этапах его обработки в общем. Важно не использовать признаки, отвечающие за то, какие этапы и через кого проходил объект.
    \item Сопоставить каждому объекту время, затраченное на его обработку.
    \item Векторизовать выбранные признаки. В данном случае достаточно закодировать через OHE.
    \item Понизить размерность и кластеризовать объекты (решил выделить 5 кластеров).
    \item В каждом кластере выбрать самые долгие по времени обработки объекты, сравнить их обработку с другими объектами.
    \item Выделить сотрудников, выполнявших эти лишние действия, и посчитать затраченное время.
\end{enumerate}

Таким образом можно выделить похожие платежи и выделить лишние действия. 

\imgh{15cm}{checks_proj.png}{Распределение чеков.}
\newpage

Лишние действия были найдены по алгоритму:
\begin{itemize}
    \item Проходим по всем кластерам, рассматриваем плохие объекты (выходящие за $3 \sigma$).

    \item По хорошим объектам ищем счастливый путь как самую частую последовательность.
    
    \item Смотрим, как путь плохих объектов отличается от счастливого и записываем лишние действия (с сотрудниками и затраченным временем).
\end{itemize}

Получил топ сотрудников, выполнявших лишние действия.

\begin{tabular}{rrr}
\hline
{} &  excessive\_time\_mean &  excessive\_stages\_count \\
excessive\_users            &                 &                   \\
\hline
Ковалева Надежда Оскаровна &     1062.090247 &                 9 \\
Белякова Антонина Юльевна  &     1055.748839 &                14 \\
Зиновьева Майя Макаровна   &      887.343207 &                11 \\
Ипатий Адрианович Ефимов   &      476.295139 &                 1 \\
Глеб Валерианович Журавлев &      325.766429 &                 2 \\
\dots & \dots & \dots \\
\hline
\end{tabular}
\newpage

\section{Поиск повторных проверок.}
Рассмотрим причины зацикливания на самых частых путях с \textit{loop\_percent} > 25. Визуализируем самые частые циклы.

\imgh{15cm}{looped_metric_graph.png}{Схема популярных циклов.}

Большинство зацикливаний происходит на этапе финальной проверки счёта. Счета отправляются на повторной проверке условий, либо передаются другому пользователю для финальной проверки.

Для оценки времени, затраченного на циклы я просто умножу коэффициент зацикленности на среднее время выполнения цепочки операций, то есть по формуле: \\
$count \times loop\_percent \times mean\_duration$

Затраченное на циклы время: 24960 дней; \\
Затраченное время на обработку чеков в общем: 163007 дней; \\
Доля времени, потраченного на циклы: 15\%.


\end{document}
